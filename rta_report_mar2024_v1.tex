% Created 2022-11-21 Mon 15:21
% Intended LaTeX compiler: pdflatex
\documentclass[11pt,twocolumn]{article}
\usepackage[utf8]{inputenc}
\usepackage[T1]{fontenc}
\usepackage{graphicx}
\usepackage{longtable}
\usepackage{wrapfig}
\usepackage{rotating}
\usepackage[normalem]{ulem}
\usepackage{amsmath}
\usepackage{amssymb}
\usepackage{capt-of}
\usepackage{hyperref}
\hypersetup{colorlinks=true,linkcolor=blue,urlcolor=blue}
\usepackage{placeins}
\input article_setup.tex
\usepackage[nottoc,numbib]{tocbibind}
\renewcommand{\UrlFont}{\small\tt}
\renewcommand*{\UrlFont}{\footnotesize}
\hypersetup{colorlinks,citecolor=black,linkcolor=blue,urlcolor=magenta}
\usepackage{ragged2e}
\date{}
\title{}
\hypersetup{
 pdfauthor={Christopher Hepplewhite},
 pdftitle={},
 pdfkeywords={},
 pdfsubject={},
 pdfcreator={Emacs 28.1 (Org mode 9.5.2)}, 
 pdflang={English}}
\begin{document}

\author{L. Larrabee Strow, Sergio DeSouza-Machado, and Chris Hepplewhite}
\date{\today}
\title{\Large Year-1/Progress Report\\ \vspace{0.05in} \normalsize Award <>
  \\The Stand Alone Rapid Transmittance Algorithm for CrIS and CHIRP}
\maketitle

\section{Overview}
\label{sec:orgc323c01}
The CrIS and CHIRP radiative transfer algorithms (RTA) are fast parameterizations
of channel-averaged atmospheric layer transmittances. The original
wrapper code is the Stand Alone Rapid Transmittance Algorithm
(SARTA) and consists of a clear sky RTA, used in generating 
L2 retrieved products from level L1C Cloud Cleared Radiances
(CCRs), and a scattering version coupled to a novel representation
of clouds as two slabs (DeSouza-Machado et al 2018).
The 2-slab model takes advantage of the fact the
hyperspectral infrared radiances have very few degrees of freedom of cloud
information, allowing one to model the radiative transfer through two
equivalent layers instead of using multiple layers and phases in the
simulation. This is used at UMBC for single footprint
retrievals at the native sensor horizontal resolution.

The SARTA transmittances are derived from simulated
transmittances computed with UMBC's pseudo line-by-line algorithm kCARTA
(kCompressed Atmospheric Radiative Transfer Algorithm), which is the
reference forward model for SARTA. The line-by-line transmittances are
convolved to the CrIS and CHIRP spectral line shapes (ILS) then fitted to a suite
of predictors used for the fast coefficients in SARTA.
The optical depths (ODs) used in kCARTA
primarily come from our custom in-house Matlab based line-by-line code
(UMBC-LBL). Additionally, kCARTA can be built with \cd and
\methane optical depths from the AER-LBLRTM code. Currently,
\methane line-mixing is not implemented in the UMBC-LBL so the AER-LBLRTM
option is used.
A complete description of both the underlying
line-by-line code (UMBC-LBL) and kCARTA can be found in
ref: DeSouza-Machado et al (2020).


\section{Year 1 Activities}
\label{sec:org92e7927}

\subsection{Variables and gases included}
\label{sec:org2c23c72}

The following is a summary list of the processes, variables and gases
that are included in the CrIS and CHIRP SARTA.
Absorption by H2O, O3, CH4, CO, HNO3, N2O, SO2, NH3, HDO, CO2 including line mixing and nonLTE,
surface TIR emissivity and solar reflectivity, including atmospheric path zenith angle
and solar zenith angle. The two-slab scattering version includes water droplet, various
ice and some dust habitats.

\subsection{Build Status at time of writing.}
\label{sec:org4d8fceb}

The current SARTA build uses the HITRAN 2020 updated line database.
The 2020 release of HITRAN was delayed significantly
to the end of 2021, this has been installed at UMBC and tested uisng the kCARTA
model. The most significant changes to the 2016 release for this application, include the
1050 cm-1 Ozone and SO2 line parameters. There are minor changes
in many other spectral regions, but after convolution to the sensor response function
the changes were generally very small. At the time of writing the new optical depths for
training profiles have been computed, from which updated fast coefficients for SARTA have mostly
been computed and tested for both the CrIS FSR and CHIRP spectral responses. The 2-slab
scattering codes will be built and tested in the next reporting period. 


\subsection{Testing, Validation and Tuning}
\label{sec:orgf433996}

At the time of this report, the testing of the current builds has consisted of quantifying
the fit residuals for the atmospheric profile training set against the kCARTA top-of-
atmosphere (TOA) brightness temperatures. There are 49 training profiles applied over a
range of view and solar angles and surface emissivities. This set of tests provides the
verification needed to demonstrate that the fast coefficient regressions have completed
successfully and where any adjustments are needed. This preceeds testing of the minor gas
perturbations, or Jacobians, and preceeds application to actual sensor records, which
in turn preceeds any tuning that may be required after comparison with conditioned tests
against radiosondes, for example.

An example of the test against kCARTA TOA BTs for the CrIS full spectral resolution (FSR)
is shown in figure 1 for various slant path angles from nadir to 60-degrees. Notice the
mean bias is typically \textasciitilde{}0.1 K with some exception in the 1300cm-1 CH4 band and N2O near 2050 cm-1.
There is very little variation with view angle, except in the 660cm-1 CO2 band. These regions
can and will be improved.

\begin{figure}[htbp]
\centering
\includegraphics[width=\linewidth]{./Figs/sar_kc_cris_hr_mean_bias_vs_angle_aslp.pdf}
\caption{\label{fig:org9a5a26c}Comparison of SARTA vs kCARTA top-of-atmosphere BT for CrIS FSR. Top: mean. Middle: mean bias. Bottom: std.dev. See text.}
\end{figure}

Figure 2 shows the difference between the current build and the previous (v2018),
which used HITRAN 2016
line parameters, calculated for CrIS FSR using the 49 atmospheric fitting profiles. The
biggest change is evident in the 1050 cm-1 Ozone band, with some small changes in other parts
of the band.

\begin{figure}[htbp]
\centering
\includegraphics[width=.9\linewidth]{./Figs/sarta_cris_hr_h2018_vs_h2020_r49_686_mn_stdaslp.pdf}
\caption{\label{fig:org538e2b1}Comparison of current and previous SARTA build for CrIS FSR. Top: Mean BT. Middle: Mean bias. Bottom: std.dev of the sample.}
\end{figure}

Figure 3 shows the CHIRP SARTA TOA BT calcluations compared to the kCARTA for the same 49 fitting
profiles and path combinations as used in figure 1 for CrIS FSR. There a similarly small bias
differences as expected, with the strong CO2 lines in the 667 cm-1 and 2350 cm-1 bands
exhibiting variation with view angle which is anticipated at this stage of
development and will be eliminated with updated training as has been done for CrIS.

\begin{figure}[htbp]
\centering
\includegraphics[width=\linewidth]{./Figs/sar_kc_chirp_mean_bias_vs_angle_aslp.pdf}
\caption{\label{fig:orgc90c8a1}Comparison of SARTA vs kCARTA top-of-atmosphere BT for CHIRP. Top: mean. Middle: mean bias. Bottom: std.dev. See text.}
\end{figure}


\subsection{Emissivity/Reflectivity Improvements}
\label{sec:orgb7d7229}

A higher fidelity model of ocean emissivity has been in review with Dr. Nalli
which includes updated surface roughness parameterization. The number of sample points
across the sensor spectral bandpass has also been increased and include the more
opaque regions outside the LW window. Studies at UMBC show that the changes are less
than about 0.1 K BT worst case. It is yet to be demonstrated that the correction improves
bias analyses, and how or if the changes will be incorporated.

In addition, the Chou emissivity scaling parameters have been under test at UMBC,
this work has just begun and an assessment will be made in the coming reporting period.


\section{Non-LTE Emission}
\label{sec:org8ad5b16}

\begin{figure}[htbp]
\centering
\includegraphics[width=\linewidth]{./Figs/predicted_nonlte_old_vs_new_vs_solzen.pdf}
\caption{\label{fig:orgd62a7e3}Old vs Extended nonLTE model.}
\end{figure}

A new set of non-LTE profiles from the atmospheric
research group (IAA) at the University of Granada, Spain were supplied
to UMBC and these have been investigated. They offer more detailed
modelling than previously and through the night time.
At the time of writing, a trial
regression has been performed to update the SARTA coefficients, and results
indicate that refinement in the choice of predictors is required. This work
will continue and the new model is planned to be incorporated subject
to being verified as improving the biases in this region, during
the next reporting period.


\section{References:}
\label{sec:orga0f8507}
DeSouza\_Machado et al. Atmos. Meas. Tech., 11, 529–550, 2018
\url{https://doi.org/10.5194/amt-11-529-2018}

DeSouza-Machado et al. Atmos. Meas. Tech., 13, 323–339, 2020
\url{https://doi.org/10.5194/amt-13-323-2020}
\end{document}